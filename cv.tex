% !TEX TS-program = luatex
% Awesome Source CV LaTeX Template
%
% This template has been downloaded from:
% https://github.com/darwiin/awesome-neue-latex-cv
%
% Author : Christophe Roger 
% Modified by : David Carayon
%
% Template license:
% CC BY-SA 4.0 (https://creativecommons.org/licenses/by-sa/4.0/)

\documentclass[localFont,alternative]{yaac-another-awesome-cv}

\name{David}{Carayon}
\photo{3.4cm}{david_square}
\tagline{Ingénieur d'études | Statisticien}
\socialinfo{
	\linkedin{davidcarayon}
	\github{davidcarayon}\\
	\smartphone{06 64 66 90 60}
	\email{carayon.david@gmail.com}\\
	\address{8 Lot l'Entrada 33650 CABANAC-ET-VILLAGRAINS}\\
	\infos{Né le 15 octobre 1994 (26 ans) à Albi (81), France}
}
%------------------------------------------
\begin{document}

\makecvheader

%% COMPETENCES
\sectionTitle{Compétences}{\faTasks}
\renewcommand{\arraystretch}{1.1}

	\begin{tabular}{>{}r>{}p{13cm}} 
		\textsc{Programmation:}              &    R (base \& tidyverse) ; Shiny avec notions en HTML/CSS  ; Git ; SQL ; Bases en python\\ 
		\textsc{Bases de données:}               	&   Collecte, nettoyage, alimentation et gestion de bases de données \\ 
		\textsc{Statistiques:}  	 &   Modélisation linéaire et non linéaire ;  Analyses multivariées ; Machine learning \\
		\textsc{Ingénierie écologique:}              &    Bioindication DCE ; gestion de la biodiversité ; suivis environnementaux \\
		\textsc{Géomatique:}			&   QGIS ; postGIS ; package \textit{sf} \\ 
		\textsc{Communication:}               	&  Suite Office ; LaTeX (Beamer) ; RMarkdown ; Dashboards 
	\end{tabular}
	
%% EXPERIENCE
\sectionTitle{Expérience professionnelle}{\faSuitcase}
%\renewcommand{\labelitemi}{$\bullet$}
\begin{experiences}
\experience
    {Aujourd'hui}   {Ingénieur d'études | Statisticien}{INRAE | UR ETTIS }{Bordeaux}
    {Juillet 2019} {
                      \begin{itemize}                    
                        \item Gestion du pipeline de traitement de la donnée, allant de la collecte à l'analyse statistique, de plusieurs projets en parallèle.
                        \item Conception de méthodes d'évaluation et des outils associés pour leur utilisation et le reporting (packages R, applications Shiny, etc.)
                        \item Conception de démarches méthodologiques et de protocoles adaptés pour l’analyse statistique de données socio-économiques et agro-environnementales
                        \item Participation à la réponse à des appels d’offre
                        \item Appui d'expertise en statistiques à l'équipe de recherche, animation autour de l'écosystème R et veille technologique                                                                         
                        \item Mise en forme et valorisation des résultats de la recherche (applications web, dashboards, rapports, articles, etc.)
                      \end{itemize}
                    }
                    {R, Shiny,SQL, Statistiques,Appui politique publique, Analyse de données}
  \emptySeparator
  \experience
    {Juin 2019}   {Ingénieur d'études | Biostatisticien en écologie aquatique}{IRSTEA | UR EABX }{Bordeaux}
    {Juillet 2017} {
                      \begin{itemize}                    
                        \item Prise en charge de la composante statistique de plusieurs projets de recherche en parallèle
                        \item Manipulation de larges jeux de données (échelle nationale) et mise en place d'algorithmes de traitement adaptés (SQL, R).
                        \item Valorisation des résultats issus des recherches via des packages open-source R (environnement de calcul d'indicateurs) ou d'autres moyen de diffusions web (cartes interactives)
                        \item Interaction permanente avec biostatisticiens, chercheurs, ingénieurs et institutions publiques (Agence Française pour la Biodiversité, offices et agences de l'eau)
                        \item Rédaction de rapports techniques, d'articles scientifiques et présentations des résultats lors de colloques internationaux
                        \item Appui d'expertise en statistiques à l'équipe de recherche, animation autour de l'écosystème R et veille technologique                                                                         
                      \end{itemize}
                    }
                    {R, SQL, Analyse de données, Appui politique publique, Bioindication, Diatomées, Recherche}
  \emptySeparator
  \experience
    {Juin 2017} {Stage de fin d'études}{DREAL Nouvelle-Aquitaine}{Bordeaux}
    {Février 2017}    {
                      \begin{itemize}              
                        \item Collecte, harmonisation et mise en base de relevés issus de 40 ans de suivis hydrobiologiques dans le cadre des réseaux nationaux de surveillance               
                        \item Analyses statistiques sur l’évolution spatio-temporelle de la structure des communautés 
                        \item Publication des résultats dans une revue scientifique à comité de lecture    
                        \item Vulgarisation des scripts et initiation du personnel à R.  
                                                
                      \end{itemize}
                    }
                    {R, Analyse de données, Animation, Invertébrés, DREAL, Bioindication}
 
  
\end{experiences}

%% FORMATION
\sectionTitle{Formation}{\faMortarBoard}

\begin{scholarship}

	\scholarshipentry{2017 - 2021}
					{Formation en ligne DataCamp : Data scientist with R (95 h) \& Statistician with R (67 h)}
	\scholarshipentry{2017}
					{Master Biodiversité et Suivis Environnementaux, Spécialité biostatistiques, de l'Université de Bordeaux}
	\scholarshipentry{2015}
					{Licence Sciences de la Vie, Mention Sciences de l'environnement, de l'Université d'Albi}
\end{scholarship}

\sectionTitle{Références}{\faQuoteLeft}
\vspace{+1em}
\textbf{Extrait :}
\begin{itemize}

\item \textbf{David Carayon}, A. Eulin Garrigue, R. Vigouroux, François Delmas. A new multimetric index for the evaluation of water ecological quality of French Guiana streams based on benthic diatoms. Ecological Indicators, Elsevier, 2020, 113, pp.10. ⟨10.1016/j.ecolind.2020.106248⟩. ⟨hal-02610306⟩
\\
\item \textbf{David Carayon}, Juliette Tison-Rosebery, François Delmas. Defining a new autoecological trait matrix for French stream benthic diatoms. Ecological Indicators, Elsevier, 2019, 103, pp.650-658. ⟨10.1016/j.ecolind.2019.03.055⟩. ⟨hal-02609271⟩
\\
\item Frédéric Zahm, J.M. Barbier, S. Cohen, H. Boureau, Sydney Girard, \textbf{David Carayon} et al.. IDEA4 : une méthode de diagnostic pour une évaluation clinique de la durabilité en agriculture. Agronomie, Environnement \& Sociétés, Association Française d'Agronomie (Afa), 2019, 9 (2), pp.39-51.
\\
\item \textbf{David Carayon} \& Guillaume Méderel (2018) Spatio-temporal evolution of benthic invertebrates communities on the Dordogne River. Ephemera 19 (1): 41‑56

\end{itemize}

\Large
\textbf{Liste complète} : \href{https://cv.archives-ouvertes.fr/david-carayon}{\color{linkcolor} {https://cv.archives-ouvertes.fr/david-carayon}}

\vspace{+1em}

\normalsize

\sectionTitle{Divers}{\faLaptop}

\begin{itemize}
\vspace{+1em}
\item Participation ponctuelle aux challenges \#TidyTuesday organisés par la communauté des statisticiens R sur Twitter\\
\item Participation active aux communautés d'entraide francophone R (groupe Slack) \\
\item Participation à de nombreuses formations et ateliers URFIST concernant la science reproductible\\
\item Hobbies : Badminton, Golf, Bricolage, Jeux

\end{itemize}
\end{document}