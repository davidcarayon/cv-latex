% !TEX TS-program = luatex
% Awesome Source CV LaTeX Template
%
% This template has been downloaded from:
% https://github.com/darwiin/awesome-neue-latex-cv
%
% Author : Christophe Roger 
% Modified by : David Carayon
%
% Template license:
% CC BY-SA 4.0 (https://creativecommons.org/licenses/by-sa/4.0/)

\documentclass[localFont,alternative]{yaac-another-awesome-cv}

\name{David}{Carayon}
\photo{3.4cm}{david_square}
\tagline{Ingénieur d'études | Biostatisticien}
\socialinfo{
	\linkedin{davidcarayon}
	\github{davidcarayon}\\
	\smartphone{+6 64 66 90 60}
	\email{carayon.david@gmail.com}\\
	\address{22 rue Louise Michel 33600 PESSAC}\\
	\infos{Né le 15 octobre 1994 (24 ans) à Albi (81), France}
}
%------------------------------------------
\begin{document}

\makecvheader

%% COMPETENCES
\sectionTitle{Compétences}{\faTasks}
\renewcommand{\arraystretch}{1.1}

	\begin{tabular}{>{}r>{}p{13cm}} 
		\textsc{Programmation:}              &    Maîtrise de R (tidyverse) ; Git (Hub/Lab) ; SQL ; Bases en python\\ 
		\textsc{Bases de données:}               	&   Collecte, nettoyage, alimentation et gestion de bases de données \\ 
		\textsc{Statistiques:}  	 &   Modélisation ;  Analyses multivariées ; Machine learning \\
		\textsc{Ingénierie écologique:}              &    Bioindication DCE ; gestion de la biodiversité ; suivis environnementaux \\
		\textsc{Géomatique:}			&   QGIS ; postGIS ; package \textit{sf} sous R \\ 
		\textsc{Communication:}               	&   LaTeX (Beamer) ; RMarkdown ; Documents interactifs (R-Shiny, Leaflet, Plotly) 
	\end{tabular}

%% EXPERIENCE
\sectionTitle{Expérience professionnelle}{\faSuitcase}
%\renewcommand{\labelitemi}{$\bullet$}
\begin{experiences}
  \experience
    {Aujourd'hui}   {Ingénieur d'études | Biostatisticien en écologie aquatique}{IRSTEA | EABX }{Bordeaux}
    {Juillet 2017} {
                      \begin{itemize}                    
                        \item Elaboration de pipeline d’analyses biomathématiques (SQL,R) en appui aux travaux sur la bioindication des  diatomées en cours d'eau
                        \item Interaction permanente avec des biostatisticiens, chercheurs, ingénieurs et des institutions publiques 
                        \item Rédaction de rapports techniques, d'articles scientifiques et présentations lors de colloques internationaux
                        \item Appui biostatistique général à l'équipe de recherche et veille technologique                                                                         
                      \end{itemize}
                    }
                    {R, SQL, Analyse de données, Appui politique publique, Bioindication, Diatomées, Recherche}
  \emptySeparator
  \experience
    {Juin 2017} {Stage en analyse de données}{DREAL Nouvelle-Aquitaine}{Bordeaux}
    {Février 2017}    {
                      \begin{itemize}              
                        \item Collecte, harmonisation et mise en base de relevés issus de 40 ans de suivis hydrobiologiques dans le cadre des réseaux nationaux de surveillance               
                        \item Analyses statistiques sur l’évolution spatio-temporelle de la structure des communautés 
                        \item Publication des résultats dans une revue scientifique à comité de lecture    
                        \item Vulgarisation des scripts et initiation du personnel à R.  
                                                
                      \end{itemize}
                    }
                    {R, Analyse de données, Animation, Invertébrés, DREAL, Bioindication}
  \emptySeparator
  \experience
    {Juillet 2016}{Stage en analyse de données}{LNHE | EDF R \& D}{Chatou(78)}
    {Mai 2016}    {
                      \begin{itemize}
                        \item Analyse statistique de la structure et de la composition des communautés de poissons en lien avec un ouvrage hydroélectrique
                        \item Analyses multivariées (ACP, AFC) et régressions (GAM) 
                        \item Rendus cartographiques (QGIS)                   
                      \end{itemize}
                    }
                    {R, QGIS,  Analyse de données, EDF, Poissons}
  \emptySeparator
  
\end{experiences}

%% LANGUES
\twocolumnsection
{\sectionTitle{Langues}{\faGlobe}
\begin{skills}
	\skill{Français}{5}
	\skill{Anglais}{5}
	\skill{Espagnol}{2}
\end{skills}}
{\sectionTitle{Forces}{\faPlus}
\vspace{1em}
\begin{itemize}
	\item Curiosité
	\item Force de proposition                    
    \item Travail en équipe 
\end{itemize}
}

%% FORMATION
\sectionTitle{Formation}{\faMortarBoard}

\begin{scholarship}

	\scholarshipentry{En cours}
					{Formation en ligne DataCamp : Data scientist with R (95 h) \& Data scientist with Python (67 h)}
	\scholarshipentry{2017}
					{Master Biodiversité et Suivis Environnementaux, Spécialité biostatistiques, de l'Université de Bordeaux}
	\scholarshipentry{2015}
					{Licence Sciences de la Vie, Mention Sciences de l'environnement, de l'Université d'Albi}
\end{scholarship}

\sectionTitle{Références}{\faQuoteLeft}

\textbf{Publications}
\begin{itemize}
\item Carayon, D. \& Méderel, G. (2018) Spatio-temporal evolution of benthic invertebrates communities on the Dordogne River. Ephemera 19 (1): 41‑56
\end{itemize}

\textbf{Communications orales}
\begin{itemize}
\item Carayon, D. \& Delmas, F. (2018). Elaborating a new autoecological trait matrix for French stream benthic diatoms. 37ème colloque de l’ADLaF, 11/09/2018-13/09/2018, Jardin botanique de Meise, Belgique.
\end{itemize}

\textbf{Rapports techniques}
\begin{itemize}
\item Guéguen, J., Carayon, D., Wach, M., Boutry, S., Coste, M., Rosebery, J., … Delmas, F. (2017). Proposed amendments to the national decree of 27 july 2015 on the methods and criteria for assessing the ecological status, the chemical status and the ecological potential of surface water (irstea sending of 19-10-2017 to the water directorate): Extracts about evolutions of contents relative to vegetal compartments in rivers of metropolitan france and overseas departments.
\end{itemize}	


\end{document}