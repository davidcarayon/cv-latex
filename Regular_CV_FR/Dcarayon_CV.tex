% !TEX TS-program = luatex
% Awesome Source CV LaTeX Template
%
% This template has been downloaded from:
% https://github.com/darwiin/awesome-neue-latex-cv
%
% Author : Christophe Roger 
% Modified by : David Carayon
%
% Template license:
% CC BY-SA 4.0 (https://creativecommons.org/licenses/by-sa/4.0/)

\documentclass[localFont,alternative]{yaac-another-awesome-cv}

\name{David}{Carayon}
\photo{3.4cm}{david_square}
\tagline{Ingénieur d'études | Statisticien}
\socialinfo{
	\linkedin{davidcarayon}
	\github{davidcarayon}\\
	\smartphone{06 64 66 90 60}
	\email{david.carayon@inrae.fr}\\
	\address{8 Lot l'Entrada 33650 CABANAC-ET-VILLAGRAINS}\\
	\infos{Né le 15 octobre 1994 (26 ans) à Albi (81), France}
}
%------------------------------------------
\begin{document}

\makecvheader

%% COMPETENCES
\sectionTitle{Compétences}{\faTasks}
\renewcommand{\arraystretch}{1.1}

	\begin{tabular}{>{}r>{}p{13cm}}  
		\textsc{Programmation:}               	&   R (avancé), Bash, LaTeX, SQL, Python \\
		\textsc{Analytiques:}              &    Modélisation statistique, Data science, Recherche reproductible, Data Visualisation\\ 
		\textsc{Statistiques:}  	 &   Analyses multivariées, Inférence, Machine learning \\
		\textsc{Packages}              &    Shiny, Tidyverse, Tidymodels, sf \\
		\textsc{Outils:}               	&  Suite Office, Rstudio, Git, Github, Docker 
	\end{tabular}
	
%% EXPERIENCE
\sectionTitle{Expérience professionnelle}{\faSuitcase}
%\renewcommand{\labelitemi}{$\bullet$}
\begin{experiences}
\experience
    {Aujourd'hui}   {Ingénieur d'études | Statisticien}{INRAE | UR ETBX }{Bordeaux}
    {Juillet 2019} {
                      \begin{itemize}                    
                        \item Gestion du pipeline de traitement de la donnée, allant de la collecte à l'analyse statistique, de plusieurs projets en parallèle.
                        \item Conception de méthodes d'évaluation et des outils associés pour leur utilisation et le reporting (packages R, applications Shiny, etc.)
                        \item Conception de démarches méthodologiques et de protocoles adaptés pour l’analyse statistique de données socio-économiques et agro-environnementales
                        \item Participation à la réponse à des appels d’offre
                        \item Appui d'expertise en statistiques à l'équipe de recherche, animation autour de l'écosystème R et veille technologique                                                                         
                        \item Mise en forme et valorisation des résultats de la recherche (applications web, dashboards, rapports, articles scientifiques à comité de lecture, etc.)
                      \end{itemize}
                    }
                    {R, Shiny,SQL,Analyse de données,Statistiques, Appui politique publique, Agriculture}
  \emptySeparator
  \experience
    {Juin 2019}   {Ingénieur d'études | Biostatisticien en écologie aquatique}{IRSTEA | UR EABX }{Bordeaux}
    {Juillet 2017} {
                      \begin{itemize}                    
                        \item Prise en charge de la composante statistique de plusieurs projets de recherche en parallèle
                        \item Manipulation de larges jeux de données (échelle nationale) et mise en place d'algorithmes de traitement adaptés (SQL, R).
                        \item Valorisation des résultats issus des recherches via des packages open-source R (environnement de calcul d'indicateurs) ou d'autres moyen de diffusions web (cartes interactives)
                        \item Interaction permanente avec biostatisticiens, chercheurs, ingénieurs et institutions publiques (Agence Française pour la Biodiversité, offices et agences de l'eau)
                        \item Rédaction de rapports techniques, d'articles scientifiques et présentations des résultats lors de colloques internationaux
                        \item Appui d'expertise en statistiques à l'équipe de recherche, animation autour de l'écosystème R et veille technologique                                                                         
                      \end{itemize}
                    }
                    {R, SQL, Analyse de données,Statistiques, Appui politique publique, Bioindication, Diatomées}
   
\end{experiences}

%% FORMATION
\sectionTitle{Formation}{\faMortarBoard}

\begin{scholarship}

	\scholarshipentry{2017 - 2021}
					{Formation en ligne DataCamp : Data scientist with R (95 h) \& Statistician with R (67 h)}
	\scholarshipentry{2017}
					{Master Biodiversité et Suivis Environnementaux, Spécialité biostatistiques, de l'Université de Bordeaux}
	\scholarshipentry{2015}
					{Licence Sciences de la Vie, Mention Sciences de l'environnement, de l'Université d'Albi}
\end{scholarship}

\sectionTitle{Divers}{\faLaptop}

\begin{itemize}
\vspace{+1em}
\item Participation ponctuelle aux challenges \#TidyTuesday organisés par la communauté des statisticiens R sur Twitter\\
\item Participation active aux communautés d'entraide francophone R (groupe Slack) \\
\item Hobbies : Badminton, Golf, Bricolage, Jeux

\end{itemize}

\end{document}