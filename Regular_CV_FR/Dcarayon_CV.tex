% !TEX TS-program = luatex
% Awesome Source CV LaTeX Template
%
% This template has been downloaded from:
% https://github.com/darwiin/awesome-neue-latex-cv
%
% Author : Christophe Roger 
% Modified by : David Carayon
%
% Template license:
% CC BY-SA 4.0 (https://creativecommons.org/licenses/by-sa/4.0/)

\documentclass[localFont,alternative]{yaac-another-awesome-cv}

\name{David}{Carayon}
\photo{3.4cm}{david_square.jpg}
\tagline{Ingénieur d'études | Statisticien}
\socialinfo{
	\linkedin{davidcarayon}
	\github{davidcarayon}\\
	\smartphone{06 64 66 90 60}
	\email{david.carayon@inrae.fr}\\
	\address{8 Lot l'Entrada 33650 CABANAC-ET-VILLAGRAINS}\\
	\infos{Né le 15 octobre 1994 (28 ans) à Albi (81), France}
}
%------------------------------------------
\begin{document}

\makecvheader

%% COMPETENCES
\sectionTitle{Compétences}{\faTasks}
\renewcommand{\arraystretch}{1.1}

	\begin{tabular}{>{}r>{}p{13cm}}  
		\textsc{Programmation:}               	&   R (+Shiny), SQL, Python, Quarto \\
		\textsc{Statistiques:}  	 &   Analyses multivariées, Inférence statistique, Machine learning \\
		\textsc{Principaux packages R}              &   Shiny, Tidyverse, data.table, Tidymodels, renv, targets, sf  \\
		\textsc{Outils:}               	&  Office, Rstudio, Git, GitHub/Lab, Docker, QGIS \\ 
		\textsc{Transversales:}              &    Gestion de projet, Recherche reproductible, Science ouverte, Data Visualisation \\
	\end{tabular}
	
%% EXPERIENCE
\sectionTitle{Expérience professionnelle}{\faSuitcase}
%\renewcommand{\labelitemi}{$\bullet$}
\begin{experiences}
\experience
    {Aujourd'hui}   {Ingénieur d'études | Statisticien}{INRAE | UR ETTIS }{Bordeaux}
    {Juillet 2019} {
                      \begin{itemize}                    
                        \item Responsable du développement des outils informatiques associés à la méthode Indicateurs de Durabilité des Exploitations Agricoles version 4 (package R, applications Shiny, base de données)
                        \item Développement de modèles prédictifs (machine learning) pour la prédiction, par exemple, du risque noyade sur le littoral girondin
                        \item Réalisation d'analyses exploratoires, descriptives puis inférentielles sur des données socio-économiques issues d'enquêtes de terrain ou de web-scrapping (ex : Twitter)
                        \item Appui méthodologique aux stagiaires et doctorants (revue de code, manipulation de données, analyses statistiques)
                        \item Rédaction ou appui aux Plans de Gestion de Données (PGD/DMP)
                        \item Pilotage de la communication interne/externe de l'unité (administration site web, pilotage chantier charte graphique, animations internes, etc.)
                      \end{itemize}
                    }
                    {R, Shiny, SQL, Statistiques, Machine learning, Agriculture, Appui politique publique, Science ouverte}
  \emptySeparator
  \emptySeparator
  \experience
    {Juin 2019}   {Ingénieur d'études | Biostatisticien}{IRSTEA | UR EABX }{Bordeaux}
    {Juillet 2017} {
                      \begin{itemize}                    
                        \item Manipulation de larges jeux de données (échelle nationale) et mise en place d'algorithmes de traitement adaptés (SQL, R).
                        \item Valorisation des résultats issus des recherches via des packages open-source R (environnement de calcul d'indicateurs) ou d'autres moyen de diffusions web (cartes interactives)
                        \item Rédaction de rapports techniques, d'articles scientifiques et présentations des résultats lors de colloques internationaux
                        \item Appui d'expertise en statistiques à l'équipe de recherche, animation autour de l'écosystème R et veille technologique                                                                         
                      \end{itemize}
                    }
                    {R, SQL, Statistiques, Appui politique publique, Hydrobiologie, Bioindication, Diatomées}
   
\end{experiences}

%% FORMATION
\sectionTitle{Formation}{\faMortarBoard}

\begin{scholarship}

	\scholarshipentry{2021}
					{Formation en ligne DataCamp : Data scientist with R (95 h) \& Statistician with R (67 h)}
	\scholarshipentry{2017}
					{Master Biodiversité et Suivis Environnementaux de l'Université de Bordeaux}
	\scholarshipentry{2015}
					{Licence Sciences de la Vie, Mention Sciences de l'environnement, de l'Université d'Albi}
\end{scholarship}

\sectionTitle{Quelques références}{\faBook}

\begin{itemize}
\vspace{+1em}
\item Participation active (développeur) du \href{https://sk8.inrae.fr}{projet INRAE inter-CATI \textcolor{symbolcolor}{SK8}} visant à proposer une plateforme d'hébergement d'applications R-Shiny sur une infrastructure Kubernetes INRAE \\
\item Code en accès libre sur \href{https://github.com/davidcarayon}{\textcolor{symbolcolor}{Github}} ou \href{https://gitlab.irstea.fr/david.carayon}{\textcolor{symbolcolor}{Gitlab}} \\

\item Production scientifique HAL : \href{https://cv.hal.science/david-carayon}{\textcolor{symbolcolor}{CV-HAL}}\\

\end{itemize}

\end{document}